\documentclass[t,12pt]{beamer}

\usepackage{pgfpages}
%\setbeameroption{show notes on second screen=left}

\usepackage{graphicx}
\usepackage{setspace}

\hypersetup{colorlinks=true,linkcolor=blue,urlcolor=blue}

\usetheme{metropolis}
\title{Introduction to \LaTeX}
\date{\today}
\author{Scrum Club}

\graphicspath{{images/}}

\begin{document}

\maketitle

%%%%%%%%%%%%%%%%%%%%%%%%%%%%%%%%%%%%%%%%%%%%%%%%%%%%%%%%%%%%%%%%%%%%%%

\begin{frame}{What is \LaTeX?}

\begin{itemize}
\item{\href{https://www.latex-project.org//}{\LaTeX} is a document preperation system}
\item{it differs from formatted text software such as Microsot Word}
\item{document is prepared in a text file with extension \texttt{.tex}}
\item{text file is compiled through an editor}
\end{itemize}

\end{frame}

%%%%%%%%%%%%%%%%%%%%%%%%%%%%%%%%%%%%%%%%%%%%%%%%%%%%%%%%%%%%%%%%%%%%%%

\begin{frame}{What are the Advantages?}

\begin{itemize}
\item{open-source}
\item{easy to get help (e.g., \href{https://tex.stackexchange.com/}{Stack Exchange})}
\item{reliable}
\item{compatible, can be run on any operating system}
\item{easy to type mathematical expressions}
\item{easy to include figures and tables}
\item{easy to manage internal references and citations}
\item{already prepared style files for specific formats}
\item{collaborative editing}
\end{itemize}

\end{frame}

%%%%%%%%%%%%%%%%%%%%%%%%%%%%%%%%%%%%%%%%%%%%%%%%%%%%%%%%%%%%%%%%%%%%%%

\begin{frame}{What Do I Need?}

\begin{itemize}
\item{a \textit{distribution} such as \href{https://miktex.org/}{MiKTeX} or \href{https://www.tug.org/texlive/}{TeX Live} (Windows users need to install this separately)}
\item{an \textit{editor} such as \href{http://www.xm1math.net/texmaker/}{TeXmaker}, \href{https://www.tug.org/texworks/}{TeXworks}, \href{https://www.texstudio.org/}{TeXstudio}}
\end{itemize}

\end{frame}

%%%%%%%%%%%%%%%%%%%%%%%%%%%%%%%%%%%%%%%%%%%%%%%%%%%%%%%%%%%%%%%%%%%%%%

\end{document}
